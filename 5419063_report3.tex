\documentclass[dvipdfmx,a4paper]{jsarticle}
\usepackage[T1]{fontenc}
\usepackage{lmodern}
\usepackage{latexsym}
\usepackage{amsfonts}
\usepackage{amssymb}
\usepackage{mathtools}
\usepackage{amsthm}
\usepackage{multirow}
\usepackage{graphicx}
\usepackage{wrapfig}
\theoremstyle{definition}
\newtheorem{thm}{定理}[section]
\newtheorem{defn}[thm]{定義}
\newtheorem{lem}[thm]{補題}
\newtheorem{prop}[thm]{命題}
\newtheorem{cor}[thm]{系}
\renewcommand{\proofname}{\textbf{証明}\nopunct}
\title{簡易的SMTPの実装}
\author{5419063 工藤正和\\日本大学文理学部情報科学科}
\date{\today}

\begin{document}

\maketitle

\begin{abstract}
  

\end{abstract}

\section{目的}
SMTP(Simple Mail Transfer Protocol)とは,電子メール転送についてのプロトコルである.
SMTPを導入する目的はRFC5321よるとメールを確実かつ効率的に転送することである.
このレポートの目的はSMTPの簡易的な実装を行い,その動作を確認することである.


\section{制作内容}
  本レポートでは,簡易的なSMTPの実装を行う.
  SMTPとは電子メール転送のプロトコルである.
  SMTPはサーバー間のメール転送だけではなく,クライアントとサーバー間でもメール転送を行う.
  前者による転送をリレー(Relay)と呼び,後者をサブミッション(Submission)と呼ぶ.
  また,リレーとサブミッションではポート番号が異なる.リレーの場合は25,サブミッションの場合は587である.
  SMTPのコマンドにはEHLO,MAIL,RCRT,DATA,RSET,VRFY,NOOP,EXPN,HELPやQUITなどが存在する.
  しかし,本実装では簡略化のため,コマンドはEHLO,MAIL,RCRT,DATAとQUITにだけ対応させる.
  また,さらなる簡略化のため,実装はサブミッションのみ行う.

\section{プログラムについて}
\subsection{プログラムの説明}
最初にSMTPクライアントのプログラムの動作を説明する.
まず,クライアントのプログラムを起動すると簡易的なコンソールが表示されるようになっている.
コンソールにコマンドを入力し,ポート番号27のSMTPサーバーにコマンドを送信する.
そして,サーバーからレスポンスが返信され,そのレスポンスが500番台のエラーならコネクションを切断する.
また,DATAコマンドを送信したとき,メール本文を入力を促す画面が表示される.
メール本文はEnterキーで一行ずつサーバーに送信される.そして,ピリオドのみを入力して送信した場合はメールの入力が終了する.
なお,サーバーとのコネクションが接続されているかどうかでも,画面表示を異なるようにしている.



次に,SMTPサーバーのプログラムの動作について説明する.
クライアントから送信された文字列がSMTPコマンド(EHLO,MAIL,RCRT,DATAとQUIT)である場合は,200,300番台のレスポンスを返す.
そうでない場合は,502  Command not implementedを返す.また,コマンドの引数が異なる場合は501  Syntax error in parameters or argumentsを返す.
その上,コマンドの順序が指定されたものと異なる場合は503  Bad sequence of commandsを返す.
なお,コマンドの順序はEHLO -> MAIL -> RCRT(複数の送信先がある場合,複数回コマンド送信してもよい) -> DATA -> QUITである.
そして,DATAコマンドが送信された時,クライアントからメール本文が送信される.QUITコマンドが送信された時,コネクションが切断される.
また,メール本文はローカルファイルに保存される.




最後に,工夫した点は次の通りである.
\begin{itemize}
  \item コマンドとコマンド引数の入力をクライアント側から大文字,小文字区別させないようにした(ドメイン名を除く)
  \item ハッシュマップでコマンド入力の順序を管理した
  \item フラグを用いてサーバに送られたデータがコマンドかメール本文かを判断した
\end{itemize}
\subsection{プログラム}
サーバーとクライアントのプログラムは次のようになっている.





\section{結論と展望}
今回はSMTPサブミッションのプロトコルを簡略化して実装を行った.
SMTPクライアントでは,SMTPコマンドとメール本文をサーバーに送信できるプログラムを作成した.
SMTPサーバーでは,クライアントからのSMTPコマンドとメール本文を処理し,様々なレスポンスを返すプログラムを作成した.
今回はサブミッションの実装だけだったが,本格的にSMTPプロトコルを実装するならば,SMTPリレーや認証の実装も必要になってくるだろう.
また,本格的なメールアプリケーションの実装する場合にはSMTPだけではなく,POPプロトコルの実装も必要になってくるだろう.

\end{document}
